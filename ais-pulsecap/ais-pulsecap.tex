\documentclass{article}
\usepackage{times}
\usepackage{amsmath}
\usepackage{graphicx}
\usepackage{geometry}

\begin{document}

\title{HV Pulse Capacitor Failure Analysis }
\author {Andrew D. Zonenberg, Ph.D \\ Research Scientist \\ Antikernel Labs}
\date{\today}
\maketitle

\section{Introduction}

Applied Ion Systems (AIS) requested a failure analysis of a Kemet C4540H683KGGWCT050 pulse capacitor, which had failed
catastrophically during maximum-power test firing of a prototype AIS-gPPT3-1C-T pulsed plasma thruster.

\begin{figure}[h]
\includegraphics[scale=0.35]{cap-failure.jpg}
\caption{Video frame from test chamber at the moment of failure}
\label{failure}
\end{figure}

Antikernel Labs performed a best-effort analysis at no cost in order to support the open-hardware nature of AIS's work.

This document is licensed under Creative Commons Attribution-NonCommercial-NoDerivatives 4.0 International.

\pagebreak
\section{Specimen Overview}

Two Kemet C4540H683KGGWCT050 capacitors were provided by AIS: the failed part, as well as an identical undamaged
capacitor removed from the failed board to be used for process development and testing.

The capacitor is a 11.4 x 10.2 x 2.7 mm multi-layer ceramic capacitor (MLCC) using C0G-type dielectric, with a nominal
capacitance of 68 nF and a 2 kV voltage rating.

\begin{figure}[h]
\includegraphics[scale=0.25]{01-goodcap-top_annotated.jpg}
\caption{Top view of undamaged capacitor}
\label{overview}
\end{figure}

\pagebreak
\section{Analysis}

The large size of the capacitor presented some unique challenges for analysis, since it was too large to fit edgewise
into most standard embedding molds.

After some experimentation with the test capacitor, the sample was temporarily embedded in transparent epoxy (AA-Bond
F110, Atom Adhesives) to prevent cracking during sawing, then sectioned parallel to the desired imaging plane with
a diamond saw. The section was then re-embedded after a 90-degree rotation

\end{document}
